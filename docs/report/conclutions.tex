La implementación de este proyecto permite la visualización gráfica del comportamiento de los distintos algoritmos de búsqueda al ser puestos a prueba en el problema objetivo. Gracias a esta visualización, se pueden analizar y comprender las diferencias y similitudes entre los algoritmos, y encontrar la estrategia de búsqueda más eficiente para el problema sugerido y para diferentes tipos de escenarios.\\

En resumen, el proyecto proporciona una plataforma para analizar y comparar los distintos algoritmos de búsqueda y determinar la estrategia de búsqueda más adecuada para el problema. Asimismo, la representación gráfica del recorrido de cada algoritmo permite comprender cómo se exploran los nodos y encontrar posibles mejoras para aumentar la eficiencia de la búsqueda.\\

Después de analizar los resultados obtenidos en el escenario objetivo, se ha llegado a la conclusión de que la búsqueda por amplitud intercalada es el algoritmo más eficiente. La razón detrás de esta elección es que, al comparar los resultados con los obtenidos por los otros dos algoritmos, se puede evidenciar que la búsqueda por amplitud intercalada ofrece una mejor eficiencia en términos de tiempo y uso de memoria.Y en su contra parte, el algoritmo menos eficiente es el algoritmo de búsqueda por profundidad iterativa debido a su estrategia de exploración y a la necesidad de realizar múltiples búsquedas en profundidad, causando así el consumo excesivo de tiempo y memoria, lo que lo vuelve un algoritmo que no garantiza una solución óptima.
