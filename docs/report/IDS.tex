El algoritmo de búsqueda por profundidad interactiva está contenido dentro de la clase \lstinline{IDS}, cuyo método constructor recibe únicamente una instancia de \lstinline{Maze} y a partir de ella define los atributos maze (matriz que representa el laberinto), \lstinline{start} (Pinocho) y \lstinline{goal} (Gepetto).\\

\begin{lstlisting}
from collections import deque
from constant import Constant
from maze import Maze



class IDS(Constant):
    """
    Class that implements the 
    Interactive-Depth search (IDS) algorithm.
    This algorithm is complete because it always finds an 
    answer (if it exists), but it does not guarantee 
    that the result is optimal.
    """

    def __init__(self, maze:Maze):
        """
        Initializes the class instance.

        Args:
            maze (Maze): Maze instance that represents the 
            board.
        """
        self.maze = maze.maze
        self.start = maze.start
        self.goal = maze.goal
\end{lstlisting}
\clearpage
La implementación del algoritmo como tal esta en el método solve() el cual implementa el algoritmo \lstinline{IDDFS.} Primero se llama al método \lstinline{_initialize()}, luego se define una variable \lstinline{depth} a cero y se inicia un ciclo while que continuará hasta que encuentre una solución o agote todas las profundidades. \\\\ En cada iteración se reinician el conjunto de visitados y la cola, y se llama al método \lstinline{_dfs()}  para buscar una solución para la profundidad actual. Si \lstinline{_dfs()} devuelve verdadero, significa que se encontró una solución y se llama al método \lstinline{_backtrack()} para recuperar el camino que lleva desde la posición de inicio hasta la meta. Si no se encontró una solución, se incrementa la profundidad en uno y se intenta de nuevo.\\\

\begin{lstlisting}
def solve(self) -> list[tuple[int, int]]:
        """
        Implements the IDDFS algorithm.

        Returns:
            (List): path from the start position to the end 
            position.
        """
        self._initialize()
        depth = 0
        while True:
            self.visited = set()
            self.queue = [self.start]
            if self._dfs(self.start, depth):
                return self._backtrack()
            depth += 1
\end{lstlisting}


\clearpage

El método \lstinline{_dfs()} implementa la búsqueda en profundidad. Recibe una posición y una profundidad máxima para explorar. Primero verifica si se alcanzó la profundidad máxima o si se llegó a la meta, y retorna verdadero en ese caso. Luego marca la posición actual como visitada y llama al método \lstinline{_explore_neighbors() } para explorar las posiciones adyacentes. Si \lstinline{_explore_neighbors() } devuelve verdadero, significa que se encontró una solución y retorna verdadero. Si no, retorna falso.\\\
    

\begin{lstlisting}
def _dfs(self, current:tuple[int,int], depth:int) -> bool:
        """
        Implements the Depth-First Search (DFS) algorithm.

        Args:
            position (Tuple): a specific position within the 
            maze.
            depth (int): the maximum depth to explore.

        Returns:
            bool: True if the goal is found.
        """
        if depth == 0:
            return False

        if self._is_goal(current):
            return True

        self._mark_visited(self.start)
\end{lstlisting}


\clearpage

A continuación, vamos a ver los métodos auxiliares que encapsulan parte de la lógica.\\\newline
El método \lstinline{_initialize()} inicializa una cola, un objeto Set (almacena elementos no repetidos) que guardará los nodos visitados, una lista de nodos visitados en orden y un diccionario \lstinline{parent} que servirá para reconstruir el camino desde la meta hasta el inicio.\\

\begin{lstlisting}
def _initialize(self):
        """
        Initializes the queue, the visited set and the 
        parent dictionary to keep track of the path.
        """
        self.queue = deque()
        self.visited = set()    # disordered, unique
        self.visited_list = []  # ordered
        self.parent = {self.start: None}
\end{lstlisting}
El método \lstinline{_is_goal()} toma una posición dentro del laberinto y evalúa si es meta, es decir, si corresponde con Gepetto.\\
\begin{lstlisting}
def _is_goal(self, position:tuple[int, int]) -> bool:
    """
    Evaluates if a given position is the goal.
    Args:
        position (tuple): a specific position within the
        maze.
    Returns:
        bool: True if the position is the goal.
    """
    return position == self.goal
\end{lstlisting}
\clearpage
El método \lstinline{_explore_neighbors() } evalúa las celdas adyacentes desde una posición dada basándose en los movimientos posibles. Luego, por cada movimiento posible se calcula la siguiente posición y si es válida, se agrega a la cola, se marca como visitada y se define como padre de la posición actual. Si la exploración de la siguiente posición lleva a una solución, retorna verdadero. Si no, continúa explorando los siguientes movimientos hasta agotarlos, y retorna falso.\\\

\begin{lstlisting}
def _explore_neighbors(self, current:tuple[int, int], 
                        depth:int) -> bool:
        """
        Evaluates neighboring cells from a given position
        based on possible moves.

        Args:
            current (tuple): a specific position within the
            maze.
        """
        # Define the possible moves: up, right, down, left
        moves = [(-1, 0), (0, 1), (1, 0), (0, -1)]

        for move in moves:
        
            next_pos = (current[0] + move[0], current[1] + 
                        move[1])
                        
            if self._is_valid_position(next_pos):
            
                self._add_to_queue(next_pos)
                self._mark_visited(next_pos)
                self._set_parent(next_pos, current)
                
                if self._dfs(next_pos, depth - 1):
                
                    return True
                    
                self.queue.pop()
                
        return False
\end{lstlisting}
\clearpage
El método \lstinline{_is_valid_position()} toma una posición (x, y) y evalúa si dicha coordenada está dentro de los límites del laberinto, si no ha sido visitada antes y si no corresponde a un muro. Esas son las condiciones necesarias para considerar una posición válida.\\
\begin{lstlisting}
def _is_valid_position(self, position:tuple[int, int])->bool:
    """
    Evaluates if the new position is within the bounds of
    the matrix and if the new position hasn't been visited
    and isn't a wall.
    Args:
        position (tuple): a specific position within the
        maze.
    Returns:
        bool: True if the position is valid, that is, it's
        inside the matrix, it hasn't been visited and it's
        not a wall.
    """
    return (
        (0 <= position[0] < self.maze.shape[0])
        and (0 <= position[1] < self.maze.shape[1])
        and position not in self.visited
        and self.maze[position] != self.WALL
    )
\end{lstlisting}
El método \lstinline{_mark_visited()} toma una posición (x, y) y la añade al conjunto y a la lista de nodos visitados. En realidad solo es necesario el conjunto visited, pero \lstinline{visited_list} lo usamos en la interfaz gráfica para ir coloreando los nodos a medida que se visitan.\\
\begin{lstlisting}
def _mark_visited(self, position:tuple[int, int]):
    """
    Evaluates if a given position has been visited.
    Args:
        position (tuple): a specific position within the
        maze.
    Returns:
        bool: True if the position has been visited.
    """
    self.visited.add(position)
    self.visited_list.append(position)

\end{lstlisting}
\clearpage
El método \lstinline{_add_to_queue()} agrega la posición a la cola.\\\
\begin{lstlisting}
def _add_to_queue(self, position:tuple[int, int]):

        """
        Adds a position to the queue.

        Args:
            position (tuple): a specific position within 
            the maze.
        """
        
        self.queue.append(position)

\end{lstlisting}
El método \lstinline{_set_parent()} toma dos nodos, primero el hijo y luego el padre, y los añade a un diccionario, siendo la clave el nodo hijo. De manera que podamos consultar en él el nodo padre de cualquier posición.\\
\begin{lstlisting}
def _set_parent(self,
    child:tuple[int, int], parent:tuple[int, int]):
    """
    Sets a position as parent of another cell in the maze.
    Args:
        child (tuple): a specific position within the maze.
        parent (tuple): a specific position within the maze.
    """
    self.parent[child] = parent
\end{lstlisting}
\clearpage
Por último, el método \lstinline{_backtrack()} reconstruye el camino desde la meta (Gepetto) hasta el nodo inicial (Pinocho). Se comienza añadiendo la meta a una lista \lstinline{path}, luego consultamos el diccionario \lstinline{parent} para obtener el padre de la meta y añadirlo a \lstinline{path} hasta llegar al inicio.
\\\newline
Si la meta no está en el diccionario, implica que el laberinto no tiene solución, por lo cual, en dicho caso se mostrará un mensaje indicando que no existe camino entre Pinocho y Gepetto.\\
\begin{lstlisting}
def _backtrack(self) -> list|str:
    """
    Backtrack from the goal to the start to find the path.
    """
    path = []
    current = self.goal

    while current is not self.start:
        path.append(current)

        # If goal is not a dict key, there is no path from
        start to goal.
        try:
            current = self.parent[current]
        except KeyError:
            return "Doesn't exist solution"

    path.append(self.start)
    #path.reverse()

    return path
\end{lstlisting}
\clearpage
