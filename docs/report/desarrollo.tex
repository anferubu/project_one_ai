Creamos una clase \lstinline{Constant}, en la que configuramos a que rol pertenece cada número, de la siguiente manera:\\
\begin{lstlisting}
class Constant:
    """
    Class that represents the numeric values of the maze.
    """

    WALL = -1
    PINOCCHIO = 0
    EMPTY = 1
    CIGAR = 2
    FOX = 3
    GEPETTO = 4
\end{lstlisting}
\clearpage
Creamos una clase \lstinline{Maze} para almacenar el tablero de juego, que se puede leer desde una matriz numérica dentro de un archivo de texto. Para esto, hacemos uso de la librería \lstinline{Numpy}, que nos permite hacer la lectura del archivo de texto y representarlo como un arreglo. Por último, se define el punto de partida (la posición de Pinocho) y el punto de destino (la posición de Gepetto).\\
\begin{lstlisting}
import numpy as np
from constant import Constant

class Maze(Constant):
    """
    Class that represents a maze.
    """

    def __init__(self, filename:str, matrix = None):
        """
        Initializes the class instance with a numeric
        matrix from a file.

        Args:
            filename (str): path to the file with the
            numeric matrix representing the maze.
        """
        if not matrix is None:
            self.maze = np.array(matrix)
        else:
            # Load maze from file.
            self.maze = np.loadtxt(filename, dtype=int)
            
        self.matrix = self.maze.tolist()

        # Define the start and goal positions.
        self.start = tuple(
            np.argwhere(self.maze == self.PINOCCHIO)[0])
        self.goal = tuple(
            np.argwhere(self.maze == self.GEPETTO)[0])
\end{lstlisting}
\clearpage