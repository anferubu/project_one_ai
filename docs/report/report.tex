\documentclass{article}
\usepackage{graphicx}
\usepackage[spanish]{babel}
\usepackage[a4paper]{geometry}
\usepackage{hyperref}
\usepackage[none]{hyphenat}
\hypersetup{
    colorlinks=true,
    linkcolor=black,
    filecolor=magenta,      
    urlcolor=cyan,
    pdfpagemode=FullScreen,
    }
\pagestyle{headings}
\title{Pinocho Univalle}
\begin{document}
\begin{titlepage}
\centering
{\includegraphics[width=0.2\textwidth]{images/logo.png}\par}
\vspace{1cm}
{\bfseries\LARGE Universidad del Valle \par}
\vspace{1cm}
{\scshape\Large Ingeniería de Sistemas \par}
\vfill
{\scshape\Huge Pinocho Univalle \par}
\vfill
{\itshape\Large Introducción a la Inteligencia Artificial \par}
\vfill
{\Large Autores: \par}
{\Large Andrés Felipe Ruíz Buriticá \par}
{\Large Kevin Steven Victoria Ospina \par}
{\Large Juan Sebastián González Camacho \par}
\vfill
{\Large \today \par}
\end{titlepage}
\tableofcontents
\clearpage
\section{Introducción}
Un agente inteligente, es una entidad capaz de percibir su entorno, procesar tales percepciones y responder o actuar en su entorno de manera racional, es decir, de manera correcta y tendiendo a maximizar un resultado esperado. Es capaz de percibir su medio ambiente con la ayuda de sensores y actuar en ese medio utilizando actuadores.
\newline \\
En este proyecto, implementaremos algunos de los algoritmos de búsqueda no informada, aplicados a los agentes inteligentes, que aprendimos en lo que lleva del curso de Introducción a la Inteligencia Artificial.
\clearpage
\section{Objetivo}
\clearpage
\section{Desarrollo}
\clearpage
\section{Conclusión}
\clearpage
\end{document}