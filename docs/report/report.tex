\documentclass[12pt, letterpaper]{article}
\usepackage{graphicx}
\usepackage{float}
\usepackage[spanish]{babel}
\usepackage[a4paper]{geometry}
\usepackage{hyperref}
\usepackage[none]{hyphenat}
\usepackage{listings}
\usepackage{color}
\hypersetup{
    colorlinks=true,
    linkcolor=black,
    filecolor=magenta,      
    urlcolor=cyan,
    pdfpagemode=FullScreen,
    }
\definecolor{deepblue}{rgb}{0,0,0.5}
\definecolor{deepred}{rgb}{0.6,0,0}
\definecolor{deepgreen}{rgb}{0,0.5,0}
\lstdefinestyle{pythonstyle}{
    language=Python,
    basicstyle=\ttfamily\small,
    morekeywords= {self},
    keywordstyle=\color{deepblue},
    stringstyle=\color{deepgreen},
    commentstyle=\color{deepgreen},
    emphstyle=\ttb\color{deepred},
    showstringspaces=false,
    breaklines=true,
    frame=single,
    breakatwhitespace=true,
}
\lstset{style=pythonstyle}
\pagestyle{headings}
\title{Pinocho Univalle}
\begin{document}
\begin{titlepage}
\centering
{\includegraphics[width=0.2\textwidth]{images/logo.png}\par}
\vspace{1cm}
{\bfseries\LARGE Universidad del Valle \par}
\vspace{1cm}
{\scshape\Large Ingeniería de Sistemas \par}
\vfill
{\scshape\Huge Pinocho Univalle \par}
\vfill
{\itshape\Large Introducción a la Inteligencia Artificial \par}
\vfill
{\Large Autores: \par}
{\Large Andrés Felipe Ruíz Buriticá \par}
{\Large Kevin Steven Victoria Ospina \par}
{\Large Juan Sebastián González Camacho \par}
\vfill
{\Large \today \par}
\end{titlepage}
\tableofcontents
\clearpage
\section{Introducción}
Un agente inteligente, es una entidad capaz de percibir su entorno, procesar tales percepciones y responder o actuar en su entorno de manera racional, es decir, de manera correcta y tendiendo a maximizar un resultado esperado. Es capaz de percibir su medio ambiente con la ayuda de sensores y actuar en ese medio utilizando actuadores.
\newline \\
En este proyecto, implementaremos algunos de los algoritmos de búsqueda no informada, aplicados a los agentes inteligentes, que aprendimos en lo que lleva del curso de Introducción a la Inteligencia Artificial.
\clearpage
\section{Objetivo}
Este proyecto tiene como objetivo aplicar los conceptos vistos en el transcurso del curso desarrollando un agente inteligente para Pinocho que le ayude a encontrar a Gepetto. Para lograrlo debemos tener en cuenta los siguientes puntos:
\begin{itemize}
    \item El agente se mueve a los espacios vacíos o hacia Gepetto con costo 1.
    \item Si el agente pasa por los cigarrillos, cuesta 2.
    \item Si el agente pasa por algún zorro, cuesta 3.
    \item La matriz debe leerse desde un archivo de texto.
    \item Se debe implementar la técnica de amplitud (en este caso aplicando una variante en la que se recorra en zig zag), costo uniforme y profundidad iterativa.
    \item Debe al final definir con buenos criterios, cuál de las tres estrategias fue mejor y por qué.
\end{itemize}
\clearpage
\section{Desarrollo}
Creamos una clase \lstinline{Constant}, en la que configuramos a que rol pertenece cada número, de la siguiente manera:\\
\begin{lstlisting}
class Constant:
    """
    Class that represents the numeric values of the maze.
    """

    WALL = -1
    PINOCCHIO = 0
    EMPTY = 1
    CIGAR = 2
    FOX = 3
    GEPETTO = 4
\end{lstlisting}
\clearpage
Creamos una clase \lstinline{Maze} para almacenar el tablero de juego, que se puede leer desde una matriz numérica dentro de un archivo de texto. Para esto, hacemos uso de la librería \lstinline{Numpy}, que nos permite hacer la lectura del archivo de texto y representarlo como un arreglo. Por último, se define el punto de partida (la posición de Pinocho) y el punto de destino (la posición de Gepetto).\\
\begin{lstlisting}
import numpy as np
from constant import Constant

class Maze(Constant):
    """
    Class that represents a maze.
    """

    def __init__(self, filename:str, matrix = None):
        """
        Initializes the class instance with a numeric
        matrix from a file.

        Args:
            filename (str): path to the file with the
            numeric matrix representing the maze.
        """
        if not matrix is None:
            self.maze = np.array(matrix)
        else:
            # Load maze from file.
            self.maze = np.loadtxt(filename, dtype=int)
            
        self.matrix = self.maze.tolist()

        # Define the start and goal positions.
        self.start = tuple(
            np.argwhere(self.maze == self.PINOCCHIO)[0])
        self.goal = tuple(
            np.argwhere(self.maze == self.GEPETTO)[0])
\end{lstlisting}
\clearpage
\subsection{Búsqueda por amplitud}
El algoritmo de búsqueda por amplitud está contenido dentro de la clase \lstinline{BFS}, cuyo método constructor recibe únicamente una instancia de \lstinline{Maze} y a partir de ella define los atributos maze (matriz que representa el laberinto), \lstinline{start} (Pinocho) y \lstinline{goal} (Gepetto).\\
\begin{lstlisting}
from collections import deque

from constant import Constant
from maze import Maze

class BFS(Constant):
    """
    Class that implements the Breadth-Preferring Search
    (BFS) algorithm.
    This algorithm is complete because it always finds
    an answer (if it exists), but it does not guarantee
    that the result is optimal.
    """

    def __init__(self, maze:Maze):
        """
        Initializes the class instance.
        Args:
            maze (Maze): Maze instance that represents
            the board.
        """
        self.maze = maze.maze
        self.start = maze.start
        self.goal = maze.goal
\end{lstlisting}
\clearpage
La implementación del algoritmo como tal está en el método \lstinline{solve()} de la misma clase. El método primero inicializa las variables necesarias, entre las cuales se tiene una cola y una lista de nodos visitados. Se toma la posición inicial, donde está Pinocho, y se añade a la cola y a la lista de nodos visitados.\\\\
Mientras la cola tenga elementos, se retira el primero de ellos y se evalúa si es la meta (Gepetto) en cuyo caso sale del ciclo. Si no es meta, se explora sus vecinos y se repite el ciclo. Una vez encontrada la meta, se traza el camino desde Gepetto hasta Pinocho.\\\\
La lógica del algoritmo está separada en distintos métodos auxiliares que facilitan la lectura del código. Posteriormente se analizará cada método utilizado en la solución.\\
\begin{lstlisting}
def solve(self) -> list|str:
    """
    Implements the BFS algorithm.This implementation
    avoids returning to already visited nodes.
    Returns:
        (List): path from the start position to the
        end position.
    """
    self._initialize()

    # Enqueue the starting position and mark it as visited.
    self._add_to_queue(self.start)
    self._mark_visited(self.start)

    while self.queue:
        current = self.queue.popleft()
        if self._is_goal(current):
            break
        self._explore_neighbors(current)
    return self._backtrack()
\end{lstlisting}
\clearpage
El algoritmo alternativo de la búsqueda por amplitud, en donde se empieza buscando de izquierda a derecha y en el siguiente nivel de derecha a izquierda, y viceversa, se implementó en el método \lstinline{solve_zigzag()}.
\\\newline
Este es muy similar al anterior, solamente que se añade una variable direction (1 o -1) para controlar el sentido de la búsqueda. Así, sí direction es 1, se saca el primer elemento de la cola, si es -1 se saca el último. Cuando se termina de explorar todos los vecinos y si aún hay elementos en la cola, \lstinline{direction} cambia de signo para comenzar a buscar en el siguiente nivel desde la dirección opuesta.
\clearpage
\begin{lstlisting}
def solve_zigzag(self) -> list|str:
    """
    Implements the BFS algorithm in a zigzag pattern. For
    this, it uses a list and takes out the nodes from the
    extreme left or right depending on the level.
    This implementation avoids returning to already visited
    nodes.
    Returns:
        (List): path from the start position to the end
        position.
    """
    self._initialize()

    # Enqueue the starting position and mark it as visited.
    self._add_to_queue(self.start)
    self._mark_visited(self.start)

    # Initialize a variable to keep track of the direction.
    # 1 for left to right, -1 for right to left.
    direction = 1

    while self.queue:
        if direction == 1:
            current = self.queue.popleft()
        else:
            current = self.queue.pop()
        if self._is_goal(current):
            break
        self._explore_neighbors(current, direction)

        # Change direction if the queue is not empty.
        if self.queue:
            direction *= -1

    return self._backtrack()
\end{lstlisting}
\clearpage
A continuación, vamos a ver los métodos auxiliares que encapsulan parte de la lógica.\\\newline
El método \lstinline{_initialize()} inicializa una cola, un objeto Set (almacena elementos no repetidos) que guardará los nodos visitados, una lista de nodos visitados en orden y un diccionario \lstinline{parent} que servirá para reconstruir el camino desde la meta hasta el inicio.\\
\begin{lstlisting}
def _initialize(self):
    """
    Initializes the queue, the visited set and the parent
    dictionary to keep track of the path.
    """
    self.queue = deque()
    self.visited = set()    # disordered, unique
    self.visited_list = []  # ordered
    self.parent = {self.start: None}
\end{lstlisting}
El método \lstinline{_is_goal()} toma una posición dentro del laberinto y evalúa si es meta, es decir, si corresponde con Gepetto.\\
\begin{lstlisting}
def _is_goal(self, position:tuple[int, int]) -> bool:
    """
    Evaluates if a given position is the goal.
    Args:
        position (tuple): a specific position within the
        maze.
    Returns:
        bool: True if the position is the goal.
    """
    return position == self.goal
\end{lstlisting}
\clearpage
El método \lstinline{_explore_neighbors()} toma una posición dentro del laberinto y el sentido de búsqueda (1 de izquierda a derecha, -1 de derecha a izquierda). El método define los posibles movimientos y su orden: arriba, derecha, abajo e izquierda.
\\\newline
Luego, para cada movimiento calcula la próxima posición y si dicha posición es válida (e.g. está dentro de los límites del laberinto) se añade dicha casilla a la lista de nodos visitados, a la cola (según la dirección dada) y al diccionario, indicando que el padre de la próxima posición es el nodo actual.\\
\begin{lstlisting}
def _explore_neighbors(self,
    current:tuple[int, int], dir:int|None=None):
    """
    Evaluates neighboring cells from a given position based
    on possible moves.
    Args:
        current (tuple): a specific position within the maze.
        dir (int): indicates the search direction.
                     1 or None: from left to right.
                    -1: from right to left.
    """
    # Define the possible moves: Up, Right, Down, Left
    moves = [(-1, 0), (0, 1), (1, 0), (0, -1)]

    for move in moves:
        # Calculates the next position according to the
        # movement.
        next_pos = (current[0] + move[0],
                    current[1] + move[1])

        if self._is_valid_position(next_pos):
            self._mark_visited(next_pos)
            self._add_to_queue(next_pos, dir)
            self._set_parent(next_pos, current)
\end{lstlisting}
\clearpage
El método \lstinline{_is_valid_position()} toma una posición (x, y) y evalúa si dicha coordenada está dentro de los límites del laberinto, si no ha sido visitada antes y si no corresponde a un muro. Esas son las condiciones necesarias para considerar una posición válida.\\
\begin{lstlisting}
def _is_valid_position(self, position:tuple[int, int])->bool:
    """
    Evaluates if the new position is within the bounds of
    the matrix and if the new position hasn't been visited
    and isn't a wall.
    Args:
        position (tuple): a specific position within the
        maze.
    Returns:
        bool: True if the position is valid, that is, it's
        inside the matrix, it hasn't been visited and it's
        not a wall.
    """
    return (
        (0 <= position[0] < self.maze.shape[0])
        and (0 <= position[1] < self.maze.shape[1])
        and position not in self.visited
        and self.maze[position] != self.WALL
    )
\end{lstlisting}
El método \lstinline{_mark_visited()} toma una posición (x, y) y la añade al conjunto y a la lista de nodos visitados. En realidad solo es necesario el conjunto visited, pero \lstinline{visited_list} lo usamos en la interfaz gráfica para ir coloreando los nodos a medida que se visitan.\\
\begin{lstlisting}
def _mark_visited(self, position:tuple[int, int]):
    """
    Evaluates if a given position has been visited.
    Args:
        position (tuple): a specific position within the
        maze.
    Returns:
        bool: True if the position has been visited.
    """
    self.visited.add(position)
    self.visited_list.append(position)

\end{lstlisting}
\clearpage
El método \lstinline{_add_to_queue()} toma una posición (x, y) y una dirección (1 o -1) y añade dicha posición a la cola según la dirección dada. Si la dirección es 1, se añade al final de la cola, si es -1 se añade al principio.\\
\begin{lstlisting}
def _add_to_queue(self,
    position:tuple[int, int], dir:int|None=None):
    """
    Adds a position to the queue.
    Args:
        position (tuple): a specific position within the
        maze.
        dir (int): indicates the search direction.
                     1 or None: from left to right.
                    -1: from right to left.
    """
    if dir == None or dir == 1:
        self.queue.append(position)
    else:
        self.queue.appendleft(position)

\end{lstlisting}
El método \lstinline{_set_parent()} toma dos nodos, primero el hijo y luego el padre, y los añade a un diccionario, siendo la clave el nodo hijo. De manera que podamos consultar en él el nodo padre de cualquier posición.\\
\begin{lstlisting}
def _set_parent(self,
    child:tuple[int, int], parent:tuple[int, int]):
    """
    Sets a position as parent of another cell in the maze.
    Args:
        child (tuple): a specific position within the maze.
        parent (tuple): a specific position within the maze.
    """
    self.parent[child] = parent
\end{lstlisting}
\clearpage
Por último, el método \lstinline{_backtrack()} reconstruye el camino desde la meta (Gepetto) hasta el nodo inicial (Pinocho). Se comienza añadiendo la meta a una lista \lstinline{path}, luego consultamos el diccionario \lstinline{parent} para obtener el padre de la meta y añadirlo a \lstinline{path} hasta llegar al inicio.
\\\newline
Si la meta no está en el diccionario, implica que el laberinto no tiene solución, por lo cual, en dicho caso se mostrará un mensaje indicando que no existe camino entre Pinocho y Gepetto.\\
\begin{lstlisting}
def _backtrack(self) -> list|str:
    """
    Backtrack from the goal to the start to find the path.
    """
    path = []
    current = self.goal

    while current is not self.start:
        path.append(current)

        # If goal is not a dict key, there is no path from
        start to goal.
        try:
            current = self.parent[current]
        except KeyError:
            return "Doesn't exist solution"

    path.append(self.start)
    #path.reverse()

    return path
\end{lstlisting}
\clearpage
\subsection{Búsqueda por costo uniforme}
Aquí va costo uniforme.
\clearpage
\subsection{Búsqueda por profundidad iterativa}
Aquí va Profundidad Iterativa.
\clearpage
\section{Conclusión}
La implementación de este proyecto permite la visualización gráfica del comportamiento de los distintos algoritmos de búsqueda al ser puestos a prueba en el problema objetivo. Gracias a esta visualización, se pueden analizar y comprender las diferencias y similitudes entre los algoritmos, y encontrar la estrategia de búsqueda más eficiente para el problema sugerido y para diferentes tipos de escenarios.\\

En resumen, el proyecto proporciona una plataforma para analizar y comparar los distintos algoritmos de búsqueda y determinar la estrategia de búsqueda más adecuada para el problema. Asimismo, la representación gráfica del recorrido de cada algoritmo permite comprender cómo se exploran los nodos y encontrar posibles mejoras para aumentar la eficiencia de la búsqueda.\\

Después de analizar los resultados obtenidos en el escenario objetivo, se ha llegado a la conclusión de que la búsqueda por amplitud intercalada es el algoritmo más eficiente. La razón detrás de esta elección es que, al comparar los resultados con los obtenidos por los otros dos algoritmos, se puede evidenciar que la búsqueda por amplitud intercalada ofrece una mejor eficiencia en términos de tiempo y uso de memoria.Y en su contra parte, el algoritmo menos eficiente es el algoritmo de búsqueda por profundidad iterativa debido a su estrategia de exploración y a la necesidad de realizar múltiples búsquedas en profundidad, causando así el consumo excesivo de tiempo y memoria, lo que lo vuelve un algoritmo que no garantiza una solución óptima.

\end{document}