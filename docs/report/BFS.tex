El algoritmo de búsqueda por amplitud está contenido dentro de la clase \lstinline{BFS}, cuyo método constructor recibe únicamente una instancia de \lstinline{Maze} y a partir de ella define los atributos maze (matriz que representa el laberinto), \lstinline{start} (Pinocho) y \lstinline{goal} (Gepetto).\\
\begin{lstlisting}
from collections import deque

from constant import Constant
from maze import Maze

class BFS(Constant):
    """
    Class that implements the Breadth-Preferring Search
    (BFS) algorithm.
    This algorithm is complete because it always finds
    an answer (if it exists), but it does not guarantee
    that the result is optimal.
    """

    def __init__(self, maze:Maze):
        """
        Initializes the class instance.
        Args:
            maze (Maze): Maze instance that represents
            the board.
        """
        self.maze = maze.maze
        self.start = maze.start
        self.goal = maze.goal
\end{lstlisting}
\clearpage
La implementación del algoritmo como tal está en el método \lstinline{solve()} de la misma clase. El método primero inicializa las variables necesarias, entre las cuales se tiene una cola y una lista de nodos visitados. Se toma la posición inicial, donde está Pinocho, y se añade a la cola y a la lista de nodos visitados.\\\\
Mientras la cola tenga elementos, se retira el primero de ellos y se evalúa si es la meta (Gepetto) en cuyo caso sale del ciclo. Si no es meta, se explora sus vecinos y se repite el ciclo. Una vez encontrada la meta, se traza el camino desde Gepetto hasta Pinocho.\\\\
La lógica del algoritmo está separada en distintos métodos auxiliares que facilitan la lectura del código. Posteriormente se analizará cada método utilizado en la solución.\\
\begin{lstlisting}
def solve(self) -> list|str:
        """
        Implements the BFS algorithm.This implementation
        avoids returning to already visited nodes.

        Returns:
            (List): path from the start position to the 
            end position.
        """
        self._initialize()

        # Enqueue the starting position and mark it 
        as visited.
        self._add_to_queue(self.start)
        self._mark_visited(self.start)

        level = 0

        while self.queue:
            current = self.queue.popleft()
            if self.levels[current] > level:
                level = self.levels[current]
            if self._is_goal(current):
                break
            self._explore_neighbors(current, level)
            print(current)
        return self._backtrack()
\end{lstlisting}
\clearpage
El algoritmo alternativo de la búsqueda por amplitud, en donde se empieza buscando de izquierda a derecha y en el siguiente nivel de derecha a izquierda, y viceversa, se implementó en el método \lstinline{solve_zigzag()}.
\\\newline
Este es muy similar al anterior, solamente que se añade una variable direction (1 o -1) para controlar el sentido de la búsqueda. Así, sí direction es 1, se saca el primer elemento de la cola, si es -1 se saca el último. Cuando se termina de explorar todos los vecinos y si aún hay elementos en la cola, \lstinline{direction} cambia de signo para comenzar a buscar en el siguiente nivel desde la dirección opuesta.\\
\begin{lstlisting}
def solve_zigzag(self) -> list|str:
        """
        Implements the BFS algorithm in a zigzag pattern. 
        For this, it uses a list and takes out the nodes 
        from the extreme left or rightdepending on the level.
        This implementation avoids returning to already
        visited nodes.
        Returns: (List): path from the start position to 
                    the end position.
        """
        self._initialize()
        self._add_to_queue(self.start)
        self._mark_visited(self.start)
        direction = 1  # 1 for left to right, -1 for right
        to left.
        level = 0
        while self.queue:
            # Determine the level of the current to know 
            its direction.
            current = self.queue[0] if direction == 1 
            else self.queue[-1]
            if self.levels[current] > level:
                level = self.levels[current]
                direction *= -1
            current = self.queue.popleft() if direction == 1 
            else self.queue.pop()
            print(f'{current} ({direction})')
            if self._is_goal(current):
                break
            self._explore_neighbors(current, level,
            direction)
        return self._backtrack()
\end{lstlisting}
\clearpage
A continuación, vamos a ver los métodos auxiliares que encapsulan parte de la lógica.\\\newline
El método \lstinline{_initialize()} inicializa una cola, un objeto Set (almacena elementos no repetidos) que guardará los nodos visitados, una lista de nodos visitados en orden y un diccionario \lstinline{parent} que servirá para reconstruir el camino desde la meta hasta el inicio.\\
\begin{lstlisting}
def _initialize(self):
        """
        Initializes the queue, the visited set and the
        parent dictionary to keep track of the path.
        """
        self.queue = deque()
        self.visited = set()    # disordered, unique
        self.visited_list = []  # ordered
        self.parent = {self.start: None}
        self.levels = {self.start: 0}
\end{lstlisting}
El método \lstinline{_is_goal()} toma una posición dentro del laberinto y evalúa si es meta, es decir, si corresponde con Gepetto.\\
\begin{lstlisting}
def _is_goal(self, position:tuple[int, int]) -> bool:
    """
    Evaluates if a given position is the goal.
    Args:
        position (tuple): a specific position within the
        maze.
    Returns:
        bool: True if the position is the goal.
    """
    return position == self.goal
\end{lstlisting}
\clearpage
El método \lstinline{_explore_neighbors()} toma una posición dentro del laberinto y el sentido de búsqueda (1 de izquierda a derecha, -1 de derecha a izquierda). El método define los posibles movimientos y su orden: arriba, derecha, abajo e izquierda.
\\\newline
Luego, para cada movimiento calcula la próxima posición y si dicha posición es válida (e.g. está dentro de los límites del laberinto) se añade dicha casilla a la lista de nodos visitados, a la cola (según la dirección dada) y al diccionario, indicando que el padre de la próxima posición es el nodo actual.\\
\begin{lstlisting}
def _explore_neighbors(self, current:tuple[int, int],
level:int|None=None, dir:int|None=None):
        """
        Evaluates neighboring cells from a given position
        based on possible moves.

        Args:
            current (tuple): a specific position within 
            the maze.
            level (int): indicates the level of the current
            node.
            dir (int): indicates the search direction.
                         1 or None: from left to right.
                        -1: from right to left.
        """
        # Define the possible moves: up, right, down, left
        if dir == None or dir == 1:
            moves = [(-1, 0), (0, 1), (1, 0), (0, -1)]
        else:
            moves = [(0, -1), (1, 0), (0, 1), (-1, 0)]

        for move in moves:
            # Calculates the next position according to the
            movement.
            next_pos = (current[0] + move[0], current[1] +
            move[1])

            if self._is_valid_position(next_pos):
                self._mark_visited(next_pos)
                self._add_to_queue(next_pos, dir)
                self._set_parent(next_pos, current)
                # Set the level of the next position.
                self.levels[next_pos] = level + 1
                
\end{lstlisting}
\clearpage
El método \lstinline{_is_valid_position()} toma una posición (x, y) y evalúa si dicha coordenada está dentro de los límites del laberinto, si no ha sido visitada antes y si no corresponde a un muro. Esas son las condiciones necesarias para considerar una posición válida.\\
\begin{lstlisting}
def _is_valid_position(self, position:tuple[int, int])->bool:
    """
    Evaluates if the new position is within the bounds of
    the matrix and if the new position hasn't been visited
    and isn't a wall.
    Args:
        position (tuple): a specific position within the
        maze.
    Returns:
        bool: True if the position is valid, that is, it's
        inside the matrix, it hasn't been visited and it's
        not a wall.
    """
    return (
        (0 <= position[0] < self.maze.shape[0])
        and (0 <= position[1] < self.maze.shape[1])
        and position not in self.visited
        and self.maze[position] != self.WALL
    )
\end{lstlisting}
El método \lstinline{_mark_visited()} toma una posición (x, y) y la añade al conjunto y a la lista de nodos visitados. En realidad solo es necesario el conjunto visited, pero \lstinline{visited_list} lo usamos en la interfaz gráfica para ir coloreando los nodos a medida que se visitan.\\
\begin{lstlisting}
def _mark_visited(self, position:tuple[int, int]):
    """
    Evaluates if a given position has been visited.
    Args:
        position (tuple): a specific position within the
        maze.
    Returns:
        bool: True if the position has been visited.
    """
    self.visited.add(position)
    self.visited_list.append(position)

\end{lstlisting}
\clearpage
El método \lstinline{_add_to_queue()} toma una posición (x, y) y una dirección (1 o -1) y añade dicha posición a la cola según la dirección dada. Si la dirección es 1, se añade al final de la cola, si es -1 se añade al principio.\\
\begin{lstlisting}
def _add_to_queue(self,
    position:tuple[int, int], dir:int|None=None):
    """
    Adds a position to the queue.
    Args:
        position (tuple): a specific position within the
        maze.
        dir (int): indicates the search direction.
                     1 or None: from left to right.
                    -1: from right to left.
    """
    if dir == None or dir == 1:
        self.queue.append(position)
    else:
        self.queue.appendleft(position)

\end{lstlisting}
El método \lstinline{_set_parent()} toma dos nodos, primero el hijo y luego el padre, y los añade a un diccionario, siendo la clave el nodo hijo. De manera que podamos consultar en él el nodo padre de cualquier posición.\\
\begin{lstlisting}
def _set_parent(self,
    child:tuple[int, int], parent:tuple[int, int]):
    """
    Sets a position as parent of another cell in the maze.
    Args:
        child (tuple): a specific position within the maze.
        parent (tuple): a specific position within the maze.
    """
    self.parent[child] = parent
\end{lstlisting}
\clearpage
Por último, el método \lstinline{_backtrack()} reconstruye el camino desde la meta (Gepetto) hasta el nodo inicial (Pinocho). Se comienza añadiendo la meta a una lista \lstinline{path}, luego consultamos el diccionario \lstinline{parent} para obtener el padre de la meta y añadirlo a \lstinline{path} hasta llegar al inicio.
\\\newline
Si la meta no está en el diccionario, implica que el laberinto no tiene solución, por lo cual, en dicho caso se mostrará un mensaje indicando que no existe camino entre Pinocho y Gepetto.\\
\begin{lstlisting}
def _backtrack(self) -> list|str:
    """
    Backtrack from the goal to the start to find the path.
    """
    path = []
    current = self.goal

    while current is not self.start:
        path.append(current)

        # If goal is not a dict key, there is no path from
        start to goal.
        try:
            current = self.parent[current]
        except KeyError:
            return "Doesn't exist solution"

    path.append(self.start)
    #path.reverse()

    return path
\end{lstlisting}
\clearpage
