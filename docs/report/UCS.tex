El algoritmo de búsqueda por costo uniforme está contenido dentro de la clase \lstinline{UCS}, cuyo método constructor recibe únicamente una instancia de \lstinline{Maze} y a partir de ella define los atributos \lstinline{maze} (matriz que representa el laberinto), \lstinline{start} (Pinocho) y \lstinline{goal} (Gepetto).\\
\begin{lstlisting}
from queue import PriorityQueue

from constant import Constant
from maze import Maze
    
class UCS(Constant):
    """
    Class that implements the Uniform-Cost Search algorithm.
    This algorithm is complete and optimal, meaning that it
    finds the shortest path if it exists.
    """
    
    def __init__(self, maze:Maze):
        """
        Initializes the class instance.
        Args:
        maze (Maze): Maze instance that represents the board.
        """
        self.maze = maze.maze
        self.start = maze.start
        self.goal = maze.goal
    
    # ...
\end{lstlisting}
\clearpage
La implementación del algoritmo como tal está en el método \lstinline{solve()} de la misma clase. El método primero inicializa las variables necesarias, entre las cuales se tiene una cola de prioridad y una lista de nodos visitados. Se toma la posición inicial, donde está Pinocho, y se añade a la cola y a la lista de nodos visitados.
\\\\
Mientras la cola tenga elementos, se retira el el elemento con la mayor prioridad (el costo más bajo) y se evalúa si es la meta (Gepetto) en cuyo caso sale del ciclo. Si no es meta, se explora sus vecinos y se repite el ciclo. Una vez encontrada la meta, se traza el camino desde Gepetto hasta Pinocho.
\\\\
La lógica del algoritmo está separada en distintos métodos auxiliares que facilitan la lectura del código. Posteriormente se analizará cada método utilizado en la solución.\\
\begin{lstlisting}
def solve(self) -> list|str:
    """
    Implements the Uniform-Cost Search algorithm.
    Returns:
        (List): path from the start position to the end
        position.
    """
    self._initialize()

    # Enqueue the starting position and mark it as visited.
    self._add_to_queue(self.start, 0)
    self._mark_visited(self.start)

    while not self.queue.empty():
        current_cost, current = self.queue.get()
        if self._is_goal(current):
            break
        self._explore_neighbors(current, current_cost)

    return self._backtrack()
\end{lstlisting}
\clearpage
A continuación, vamos a ver los métodos auxiliares que encapsulan parte de la lógica.
\\\\
El método \lstinline{_initialize()} inicializa una cola de prioridad, un objeto \lstinline{Set} (almacena elementos no repetidos) que guardará los nodos visitados, una lista de nodos visitados en orden, un diccionario parent que servirá para reconstruir el camino desde la meta hasta el inicio y un diccionario que almacenará el costo acumulado de cada nodo.\\
\begin{lstlisting}
def _initialize(self):
    """
    Initializes the queue, the visited set and the parent
    dictionary to keep track of the path.
    """
    self.queue = PriorityQueue()
    self.visited = set()
    self.visited_list = []
    self.parent = {self.start: None}
    self.cost_so_far = {self.start: 0}

\end{lstlisting}
El método \lstinline{_is_goal()} toma una posición dentro del laberinto y evalúa si es meta, es decir, si corresponde con Gepetto.\\
\begin{lstlisting}
def _is_goal(self, position:tuple[int, int]) -> bool:
    """
    Checks if a position is the goal.
    Args:
        position (tuple): a specific position within the
        maze.
    Returns:
        (bool): True if the position is the goal, False
        otherwise.
    """
    return position == self.goal

\end{lstlisting}
\clearpage
El método \lstinline{_explore_neighbors()} toma una posición dentro del laberinto y el costo acumulado en ese nodo. El método define los posibles movimientos y su orden: arriba, derecha, abajo e izquierda.
\\\\
Luego, para cada movimiento calcula la próxima posición y si dicha posición es válida (e.g. está dentro de los límites del laberinto) se calcula el nuevo costo y se añade la tupla (posición, costo acumulado) a la cola de prioridad, se añade el nodo a la lista de visitados y se actualizan los diccionarios parent y \lstinline{cost_so_far}.\\
\begin{lstlisting}
def _explore_neighbors(self,
    current:tuple[int, int], current_cost:int):
    """
    Evaluates neighboring cells from a given position based
    on possible moves.
    Args:
        current (tuple): a specific position within the maze.
        current_cost (int): the cost of the current path to
        reach this position.
    """
    # Define the possible moves: Up, Right, Down, Left
    moves = [(-1, 0), (0, 1), (1, 0), (0, -1)]

    for move in moves:
            # Calculates the next position according to the
            movement.
            next_pos = (current[0] + move[0],
                        current[1] + move[1])

            if self._is_valid_position(next_pos):
                # Calculate the cost of the new path and add
                it to the queue.
                new_cost = current_cost + self.maze[next_pos]
                self._add_to_queue(next_pos, new_cost)

                # Mark the cell as visited.
                self._mark_visited(next_pos)

                # Update the parent dictionary and the
                # cost_so_far dictionary.
                self._set_parent(next_pos, current)
                self.cost_so_far[next_pos] = new_cost


\end{lstlisting}
\clearpage
El método \lstinline{_is_valid_position()} toma una posición (x, y) y evalúa si dicha coordenada está dentro de los límites del laberinto, si no ha sido visitada antes y si no corresponde a un muro. Esas son las condiciones necesarias para considerar una posición válida.\\
\begin{lstlisting}
def _is_valid_position(self,
    position:tuple[int, int]) -> bool:
    """
    Checks if a position is within the maze and is not an
    obstacle.
    Args:
    position (tuple): a specific position within the maze.
    Returns:
        (bool): True if the position is valid, False
        otherwise.
    """
    return (
        (0 <= position[0] < self.maze.shape[0])
        and (0 <= position[1] < self.maze.shape[1])
        and position not in self.visited
        and self.maze[position] != self.WALL
    )

\end{lstlisting}
El método \lstinline{_mark_visited()} toma una posición (x, y) y la añade al conjunto y a la lista de nodos visitados. En realidad solo es necesario el conjunto \lstinline{visited}, pero \lstinline{visited_list} lo usamos en la interfaz gráfica para ir coloreando los nodos a medida que se visitan.\\
\begin{lstlisting}
def _mark_visited(self, position:tuple[int, int]):
    """
    Marks a position as visited.
    Args:
        position (tuple): a specific position within the
        maze.
    """
    self.visited.add(position)
    self.visited_list.append(position)
\end{lstlisting}
\clearpage
El método \lstinline{_add_to_queue()} toma una posición (x, y) y un costo y los añade ambos a la cola.\\
\begin{lstlisting}
def _add_to_queue(self, position:tuple[int, int], cost:int):
    """
    Adds a position to the queue.
    Args:
        position (tuple): a specific position within the
        maze.
        cost (int): the cost of the current path to reach
        this position.
    """
    self.queue.put((cost, position))
\end{lstlisting}
El método \lstinline{_set_parent()} toma dos nodos, primero el hijo y luego el padre, y los añade a un diccionario, siendo la clave el nodo hijo. De manera que podamos consultar en él el nodo padre de cualquier posición.\\
\begin{lstlisting}
def _set_parent(self,
    child:tuple[int, int], parent:tuple[int, int]):
    """
    Sets the parent of a given position.
    Args:
        child (tuple): a specific position within the maze.
        parent (tuple): a specific position within the maze.
    """
    self.parent[child] = parent
\end{lstlisting}
\clearpage
Por último, el método \lstinline{_backtrack()} reconstruye el camino desde la meta (Gepetto) hasta el nodo inicial (Pinocho). Se comienza añadiendo la meta a una lista \lstinline{path}, luego consultamos el diccionario parent para obtener el padre de la meta y añadirlo a \lstinline{path} hasta llegar al inicio.
\\\\
Si la meta no está en el diccionario, implica que el laberinto no tiene solución, por lo cual, en dicho caso se mostrará un mensaje indicando que no existe camino entre Pinocho y Gepetto.\\
\begin{lstlisting}
def _backtrack(self) -> list|str:
    """
    Backtracks from the goal position to the start position.
    Returns:
    (List): path from the start position to the end position.
    """
    path = []
    current = self.goal

    while current is not self.start:
        path.append(current)

        # If goal is not a dict key, there is no path from
        # start to goal.
        try:
            current = self.parent[current]
        except KeyError:
            return "Doesn't exist a solution"

    path.append(self.start)
    #path.reverse()

    return path

\end{lstlisting}

\clearpage